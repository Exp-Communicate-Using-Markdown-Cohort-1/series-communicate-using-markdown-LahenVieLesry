\documentclass[journal]{IEEEtran}

\ifCLASSINFOpdf
\else
   \usepackage[dvips]{graphicx}
\fi
\usepackage{url}

\hyphenation{op-tical net-works semi-conduc-tor}

\usepackage{graphicx}


\begin{document}

\title{Selective Audio Adversarial Example Hidden Information on Multilingual Speech Recognition System}

\author{First A. Author, \IEEEmembership{Fellow, IEEE}, Second B. Author, and Third C. Author, Jr., \IEEEmembership{Member, IEEE}
\thanks{This paragraph of the first footnote will contain the date on which you submitted your paper for review. It will also contain support information, including sponsor and financial support acknowledgment. For example, ``This work was supported in part by the U.S. Department of Commerce under Grant BS123456.'' }
\thanks{The next few paragraphs should contain the authors' current affiliations, including current address and e-mail. For example, F. A. Author is with the National Institute of Standards and Technology, Boulder, CO 80305 USA (e-mail: author@boulder.nist.gov).}
\thanks{S. B. Author, Jr., was with Rice University, Houston, TX 77005 USA. He is now with the Department of Physics, Colorado State University, Fort Collins, CO 80523 USA (e-mail: author@lamar.colostate.edu).}}

\markboth{Journal of \LaTeX\ Class Files, Vol. 14, No. 8, August 2015}
{Shell \MakeLowercase{\textit{et al.}}: Bare Demo of IEEEtran.cls for IEEE Journals}
\maketitle

\begin{abstract}
Given the extensive research and practical applications of Automatic Speech Recognition (ASR), there has been an increasing interest in realizing adversarial attacks on ASR systems by injecting small perturbations. Previous explorations of adversarial attacks on ASR systems have focused on a single language and a single model, causing the ASR system to transcribe the wrong speech content. However, with the development of multilingual ASR systems, adversarial samples can enable strongly covert message transmission tasks. For example, when adversarial samples are applied to eavesdropping devices in military environments, friendly eavesdropping devices can recognize the hidden secret messages in the adversarial samples, while enemy devices (even if the models are identical) cannot access the hidden secret messages. We thoroughly investigate the properties of multilingual ASR systems and propose a selective audio adversarial sample that is injected with a minimal perturbation hidden secret information will be accurately extracted by friendly classifiers but correctly classified as original phrases by enemy classifiers. Our comprehensive experimental results show highly successful results achieved in three models on two datasets.
\end{abstract}

\begin{IEEEkeywords}
Adversarial example, machine learning, multilingual speech recognition.
\end{IEEEkeywords}


\IEEEpeerreviewmaketitle



\section{Introduction}

\section{PRELIMINARIES}

\section{PROPOSED METHOD}
\subsection{Problem Formulation}
A target adversarial attack to the multilingual speech recognition System 

\section{EXPERIMENTS}

\section{CONCLUSION}

\end{document}
